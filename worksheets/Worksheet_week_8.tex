\documentclass[12pt]{article}
\usepackage{amsthm,amsmath,amsfonts,amsthm,amstext,amssymb,fullpage,framed,fancybox,graphicx,color,mdwlist,pifont, hyperref,enumitem}
%\usepackage{fullpage}
\usepackage{tikz}
\def\checkmark{\tikz\fill[scale=0.4](0,.35) -- (.25,0) -- (1,.7) -- (.25,.15) -- cycle;} 
\usepackage[margin=1cm]{geometry}
\pagestyle{empty}
\newtheorem*{theorem*}{Theorem}

\newcommand{\set}[1]{\left\{ #1 \right\}}
\renewcommand{\neg}{\sim}
\newcommand{\st}{\text{ s.t. }}
\begin{document}
\centerline{\bf\large Worksheet for Week 8}

\vspace{25pt}



\begin{enumerate}

\item Let $A$, $B$ and $C$ be sets. Then prove or disprove: $A\cap (B\cup C)=(A\cap B)\cup (A\cap C)$.

%\textbf{Proof:} Since this is a set equality, we need to prove $A\cap (B\cup C)\subseteq(A\cap B)\cup (A\cap C)$ and
%
% $(A\cap B)\cup (A\cap C)\subseteq A\cap (B\cup C)$.
% 
% \textbf{Proof of $A\cap (B\cup C)\subseteq(A\cap B)\cup (A\cap C)$:} Assume that $x\in A\cap (B\cup C)$. Then we know that $x\in A$ and $x\in (B\cup C)$. That means $x\in A$ and moreover, $x\in B$ or $x\in C$. Then we have two cases,  $x\in A$ and $x\in B$ or $x\in A$ and $x\in C$. First assume $x\in A$ and $x\in B$. Thus, $x\in (A\cap B)$. This implies $x\in (A\cap B)\cup (A\cup C)$. We see that the case $x\in A$ and $x\in C$ is treated similarly to the previous case. Therefore, $A\cap (B\cup C)\subseteq(A\cap B)\cup (A\cap C)$ 
% 
% \textbf{Proof of  $(A\cap B)\cup (A\cap C)\subseteq A\cap (B\cup C)$:} Let $y\in (A\cap B)\cup (A\cap C)$. Then we have two cases: $y\in (A\cap B)$ or $y\in (A\cap C)$. Since these cases are similar, WLOG we can assume $y\in (A\cap B)$. Then we see that $y\in A$ and $y\in B$. Thus, we see $y\in A$ and $y\in B\cup C$. Therefore $y\in A\cap (B\cup C)$. Hence, $(A\cap B)\cup (A\cap C)\subseteq A\cap (B\cup C)$.
% 
% Therefore, we see that for  $A$, $B$ and $C$ be sets. Then $A\cap (B\cup C)=(A\cap B)\cup (A\cap C)$.

\item Let $A$, $B$ and $C$ be sets. Then prove or disprove: $A-(B-C)=(A-B)-C$.

Hint: Venn diagrams will be useful here.

\item If $A$ and $B$ are sets, then prove that $(A\subseteq B)\implies (\mathcal P(A)\subseteq \mathcal P(B))$.

Hint: Remember that subset relation is a conditional statement.

\item (Old final question) Let $A,B,C$ be sets. Prove that $A\times C\subseteq B\times C$ if and only if $A\subseteq B$ or $C=\emptyset$.

Hint: For one of the directions, recall $\left(P\implies (Q\lor R)\right)\equiv \left(P\implies ((\neg Q)\implies R)\right)$.

%{\bf Proof:}  Let $A,B,C$ be sets.
%
%{\bf Show ``$A\subseteq B$ or $C=\emptyset$ implies $A\times C\subseteq B\times C$'': } Assume that $A\subseteq B$ or $C=\emptyset$. Then, we have $2$ cases.
%
%{\bf Case 1: $A\subseteq B$:} Let $(a,c)\in A\times C$. Then, we know that $a\in A$ and $c\in C$. Since $A\subseteq B$, we see that $a\in B$. Thus, $(a,c)\in B\times C$. Hence $A\times C\subseteq B\times C$.
%
%{\bf Case 2: $C=\emptyset$:} In this case, we see that $A\times C=\emptyset=B\times C$.
%
%Therefore, in both cases, $A\times C\subseteq B\times C$.
%
%{\bf Show `` $A\times C\subseteq B\times C$ implies $A\subseteq B$ or $C=\emptyset$ '': } Assume that $A\times C\subseteq B\times C$, and that $C\neq \emptyset$, since otherwise the statement is trivially true. 
%
%Now, let $a\in A$. Then, since $C\neq \emptyset$, we see that there exists $c\in C$, and thus, $(a,c)\in A\times C$. Moreover, since $A\times C\subseteq B\times C$, we see $(a,c)\in B\times C$. This implies that $a\in B$.
%
%Thus, $A\subseteq B$.
%
%Therefore, for sets $A, B, C$, $A\times C\subseteq B\times C$ if and only if $A\subseteq B$ or $C=\emptyset$.

\item If $A$ and $B$ are sets, then prove that $\mathcal{P}(A)\cap\mathcal{P}(B) = \mathcal{P}(A\cap B)$.

%\textbf{Proof:} This is a set equality, so we have to prove $\mathcal{P}(A)\cap\mathcal{P}(B) \subseteq \mathcal{P}(A\cap B)$ and $ \mathcal{P}(A\cap B)\subseteq \mathcal{P}(A)\cap\mathcal{P}(B)$.
%
%\textbf{Proof of $\mathcal{P}(A)\cap\mathcal{P}(B) \subseteq \mathcal{P}(A\cap B)$:} Let $X\in \mathcal{P}(A)\cap\mathcal{P}(B) $. Then we see that $X\in\mathcal{P}(A)$ and $X\in \mathcal{P}(B)$. This implies that $X\subseteq A$ and $X\subseteq B$. Then we see that if $z\in X$, then $z\in A$ and $z\in B$, that is $z\in A\cap B$. Hence, $X\subseteq A\cap B$. Then we see that $X\in \mathcal{P}(A\cap B)$.
%
%\textbf{Proof of  $ \mathcal{P}(A\cap B)\subseteq \mathcal{P}(A)\cap\mathcal{P}(B)$:} Let $Y\in \mathcal{P}(A\cap B)$. Then we see that $Y\subseteq A\cap B$. Thus, if $t\in Y$. then $t\in A\cap B$. This implies $t\in A$ and $t\in B$. Thus, we see that $Y\subseteq A$ and $Y\subseteq B$. This means that $Y\in  \mathcal{P}(A)$ and $Y\in \mathcal{P}(B)$. Therefore $Y\in \mathcal{P}(A)\cap\mathcal{P}(B)$.
%
%Hence, the result follows.
\end{enumerate}

Before the following examples, watch videos 26 and 27 in \url{https://personal.math.ubc.ca/~PLP/auxiliary.html}.

\begin{enumerate}[resume]
\item Let $A =\{1,2,3,4,5,6\}$. Write out the relation $R$ that expresses ``$\nmid$'' (does not divide) on $A$ as a set of ordered pairs.



Note: We need to make sure that we are not leaving any one of the ordered pairs out. If we do, it is not the same relation anymore.

\item (Old final) Determine which of the following relations, $\mathbf{R}$, are reflexive, symmetric and transitive on the given set $A$. (We call a relation that satisfies all 3 properties, an equivalence relation.) Prove your answers.

\begin{enumerate}

\item  $\mathbf{R}=\set{(x,y)\in\mathbb R\times \mathbb R\colon y=x^2}$ on $A=\mathbb R$.

\item  $\mathbf{R}=\set{(1,1),(2,2),(3,3),(1,2),(2,1),(1,3),(3,1)}$ on $A=\set{1,2,3}$.

\item  $\mathbf{R}=\set{(a,b)\in\mathbb Z\times \mathbb Z\colon 3 \text{ divides } a-b}$ on $A=\mathbb Z$.

\end{enumerate}

%\textbf{Solution:}
%
%\begin{enumerate}
%
%\item We see that this relation is not reflexive since $(2,2)\notin \mathbf{R}$, that is, $\mathbf{R}$ (students may also mention that other properties fail too).
%
%\item We see that this relation is not an equivalence relation since $(2,1), (1,3)\in \mathbf{R}$, but $(2,3)\notin \mathbf{R}$, that is, $\mathbf{R}$ is not transitive.
%
%\item We see that this relation is reflexive since for any $a\in\mathbb Z$, we see $3\mid 0=(a-a)$. 
%
%We also see that this relation is symmetric since if $a,b \in\mathbb Z$ and $3\mid (a-b)$, then we know that $a-b=3k$ for some $k\in\mathbb Z$ and thus, $b-a=3(-k)$. Therefore, since $-k\in\mathbb Z$, we see that $3\mid (b-a)$, that is $b\mathbf{R}a$. 
%
%This relation is also transitive since if $a,b,c\in\mathbb Z$ and $3\mid (a-b)$ and $3\mid (b-c)$, then we see $a-b=3k$ and $b-c=3m$ for some $k,min\mathbb Z$. Thus, $a-c=(a-b)+(b-c)=3(k+m)$, which implies $3\mid (a-c)$, since $k+m\in\mathbb Z$. Therefore $a\mathbf{R}c$.
%
%Hence, $\mathbf{R}$ is an equivalence relation.
%
%\end{enumerate}


\item Define a relation on $\mathbb Z$ as $a R b$ if $3\mid (2a-5b)$. Is $R$ reflexive, symmetric, transitive? Justify your answer.

%\textbf{Solution:} We need to check whether this relation is reflexive, symmetric, and transitive.
%\begin{itemize}
%\item[] \textbf{Reflexive:} We see that this relation is reflexive since for any $a\in\mathbb Z$, we have $(2a-5a)=3(-a)$, which implies $3\mid (2a-5a)$, that is, $aRa$. 
%\item[] \textbf{Symmetric:} Let $a,b\in\mathbb Z$ and assume $aRb$. Then we see $3\mid (2a-5b)$, and so $2a-5b=3k$ for some $k\in\mathbb Z$. Then $2b-5a = (-3b-3a)-(2a-5b) = 3(-b-a-k)$. Since $(-b-a-k)\in\mathbb{Z}$ we see that $3\mid (2b-5a)$. Therefore $R$ is symmetric.
%\item[] \textbf{Transitive:} Let $a,b,c\in\mathbb Z$ and assume $aRb$ and $bRc$. Then we see $3\mid (2a-5b)$ and $3\mid (2b-5c)$, so that $2a-5b=3k$ and $2b-5c=3n$ for some $k,n\in\mathbb Z$. Then $2a-5c = (2a-5b)+3b+(2b-5c) = 3(k+b+n)$. Since $(k+b+n)\in\mathbb{Z}$ we see that $3\mid (2a-5c)$. Therefore $R$ is transitive.
%\end{itemize}

\item Let $\mathcal R$ and $\mathcal R'$ be two relations on the same set $A$. Prove or disprove the following.
\begin{enumerate}
\item If $\mathcal R$ and $\mathcal R'$ are transitive, then the relation $\widehat{\mathcal R}$ defined as $\widehat{\mathcal R}=\mathcal R \cup\mathcal R'$ is transitive.
\item If $\mathcal R$ and $\mathcal R'$ are transitive, then the relation $\widehat{\mathcal R}$ defined as $\widehat{\mathcal R}=\mathcal R \cap\mathcal R'$ is transitive.
%\item If $\mathcal R$ and $\mathcal R'$ are symmetric, then $\mathcal R \cup\mathcal R'$ is symmetric.
%\item If $\mathcal R$ and $\mathcal R'$ are symmetric, then $\mathcal R \cap\mathcal R'$ is symmetric.
\end{enumerate}

\end{enumerate}

Before the following week, watch video 28 in \url{https://personal.math.ubc.ca/~PLP/auxiliary.html}.


\end{document} 