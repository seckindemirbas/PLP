\documentclass[12pt]{article}
\usepackage{amsthm,amsmath,amsfonts,amsthm,amstext,amssymb,fullpage,framed,fancybox,graphicx,color,mdwlist,pifont, hyperref}
%\usepackage{fullpage}
\usepackage{tikz}
\def\checkmark{\tikz\fill[scale=0.4](0,.35) -- (.25,0) -- (1,.7) -- (.25,.15) -- cycle;} 
\usepackage[margin=1cm]{geometry}
\pagestyle{empty}
\newtheorem*{theorem*}{Theorem}

\newcommand{\set}[1]{\left\{ #1 \right\}}
\begin{document}
\centerline{\bf\large Worksheet for Week 1}

\vspace{25pt}

Before the first class watch the videos 1,2, and 3 in \url{https://personal.math.ubc.ca/~PLP/auxiliary.html}

Examples to cover in class

\begin{enumerate}

\item List the elements of the following sets.

\begin{itemize}

\item $\set{n^2 \mid n\in\mathbb Z}$

\item $\set{x\in\mathbb Z \mid x^2-2=0}$

\item $\set{x\in\mathbb Z \mid x^2-2=0}$

\item $\set{x^2\mid x\in (-3, 1]}$

\item $B=\set{x\in A\mid x<\frac{1}{2\pi}}$ given that  $A=\set{\frac{1}{n}\mid n\in\mathbb N}$.

\end{itemize}


\item  Let $A=\set{\ldots, -8, -4, 0, 4, 8, \ldots}$ and $B=\set{\ldots, -6, -3, 0, 3, 6, \ldots}$.  Write the sets $A$ and $B$ in set builder notation.

Now, let $C$ be the set of all elements which are sums of an element from  $A$ and an element from $B$. Write $C$ in a set builder notation in two different ways.

What can we say about the elements of this set? Is $1\in C$, $2\in C$, $3\in C$, $5\in C$? Discuss.

\item Consider the following problem:

Assume that we got rid of all currency and introduced new coins worth $3$ `stones' and $7$ `stones' ({\it stone} is our new currency now). Also assume that everyone has enough coins. Write in set-builder notation the set of prices an object can be charged.

(E.g: If we only had coins worth $4$ and $19$ stones, we could still charge $1$ stone, since someone could give us $5$ coins of $4$ and we could return $1$ coin of $19$).


\end{enumerate}

Before the next lecture, watch  videos 4,5, and 7 in \url{https://personal.math.ubc.ca/~PLP/auxiliary.html}



\end{document} 